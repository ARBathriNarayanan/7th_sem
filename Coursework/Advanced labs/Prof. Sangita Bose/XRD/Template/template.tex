\documentclass[]{report}[12 pt]
\usepackage{geometry}
\usepackage{amsmath}
\usepackage{graphicx}
\usepackage{hyperref}
\geometry{margin= 1.5 cm}
\begin{document}
	\begin{titlepage}
	\begin{center}
		\vspace*{1cm}
		
		\Huge
		\textbf{Laboratory Report}
		
		\vspace{0.5cm}
		\LARGE
		Microwave Plasma Chemical Wave Deposition\\
		\vspace{0.5cm}
		\textbf{Guide: Prof. Padmnabh Rai}
		
		\vspace{1.5cm}
		
		\textbf{A R Bathri Narayanan}\\
		Roll no: P0211501\\
		UM DAE Centre for Excellence in Basic Sciences
		
		\vspace{3 cm}
		
		Report presented for the\\
		Advanced Physics Laboratory Course (PL 701)
		
		\vspace{0.8cm}
		
		\includegraphics[width=0.4\textwidth]{cebs.jpg}
		
		\Large
		School of Physical Sciences\\
		UM-DAE Centre for Excellence in Basic Sciences\\
		Mumbai, MH, India\\
		\today
		
	\end{center}
\end{titlepage}
	\section*{Objectives:}
	\begin{enumerate}
	\item To study the theory of X Ray diffraction, and use it on the given crystal (Lithium Fluoride)
	\item To use different X-Rays (Copper and Molybdenum) and obtain the spectrum and hence the wavelength
	\item To use a filter (Nickel for Copper and Zirconium for Molybdenum) and find out it's effects on the spectrum
	\item To find out the structure of the given material (Copper).
	\end{enumerate}
	\section*{Apparatus required}
	X Ray diffractometer, a crystal of Li-F, Cu and Mo metal targets, Computer for analysis and measuring.
	\section*{Theory}
	X Ray diffraction is a technique used to find atomic and molecular structures of a crystal. The atoms of a crystal, by virtue of their uniform spacing, cause an interference pattern of the waves present in an incident beam of X-rays. 
	
	\section*{Observations}
	\subsection*{Target:Copper, Crystal:LiF, Filter: None}
	We obtain a spectrum like this\\
	\begin{center}
		\includegraphics[width=10 cm]{Cu Target, LiF crystal, No Filter.png}\\
		\textit{Spectrum of LiF XRD with Copper Target and no filter}
	\end{center}
	We try to match it with the table given to us

		\begin{center}
		\begin{tabular}{|c|c|c|c|c|}
		\hline
		2 $\theta$ & Intensity & h & k & l \\
		\hline
		38.696 & 95 & 1 & 1 & 1 \\
		\hline
		44.996 & 100 & 2 & 0 & 0 \\
		\hline
		65.494 & 48 & 2 & 2 & 0 \\
		\hline
		78.765 & 10 & 3 & 1 & 1 \\
		\hline
		82.998 & 11 & 2 & 2 & 2 \\
		\hline
		99.628 & 3 & 4 & 0 & 0 \\
		\hline
		112.967 & 4 & 3 & 3 & 1 \\
		\hline
		117.606 & 14 & 4 & 2 & 0 \\
		\hline
		139.134 & 13 & 4 & 2 & 2 \\
		\hline
	\end{tabular}\\
	\textit{2$\theta$ vs Intensity data for Lithium Fluoride}
	\end{center}
	At the first thought, it seems like the 38.696, 44.996, to an extent 82.998 and 99.628 bands are visible. So we mark that.
	\begin{center}
		\includegraphics[width=10 cm]{a1.png}\\
		\textit{Marked spectrum of LiF XRD with Copper Target and no filter}
	\end{center}

\subsection*{Target:Copper, Crystal:LiF, Filter:Ni}
We obtain a spectrum like this\\
\begin{center}
	\includegraphics[width=10 cm]{Cu Target, LiF crystal, Ni Filter.png}\\
	\textit{Spectrum of LiF XRD with Copper Target and Ni filter}
\end{center}
This comes as a surprise for us as
\begin{enumerate}
	\item There is no (1 1 1) band or (2 2 2) band in the filtered one
	\item There are bands of only one kind predominant, that is the (h,0,0) type bands. We present the combined two graphs together for a better inference
		\begin{center}
		\includegraphics[width=10 cm]{comb1.png}\\
		\textit{Combined spectrum of LiF XRD and Copper Target, with and without filter}
	\end{center}
\end{enumerate}

\subsection*{Inferences}
\begin{itemize}
	\item The presence of just (h,0,0) bands should imply the crystal lattice is oriented in a single direction, hence it is a mono-crystalline solid.
	\item We can find out the wavelength of the X-Ray with this spectrum.
	We know\\
	\begin{equation*}
		2dsin\theta=n\lambda
	\end{equation*}
	We have 
	\[d=\frac{a}{\sqrt{h^2+k^2+l^2}}\]
	We have a to be 402 pm or 4.02 {\AA}  which is the lattice constant of LiF.
	Calculating the $\lambda$ of the X-Ray, we get
	\begin{align*}
		\lambda &= \frac{2\times4.02\times10^{-10}}{\sqrt{2^2+0^2+0^2}} sin(44.8/2)\\
		\lambda &= 1.531 {\AA}
	\end{align*}
	The uncertainty stems up from the least count of the machine (0.2 degrees). Accounting that\\
	\textbf{$\lambda$ = 1.531 $\pm$ 0.007 \AA}
		\item So the line which we incorrectly marked as (1 1 1) was actually the $K_{\beta}$ line, the major one being the $K_{\alpha}$ line. Doing the similar calculations for the $K_{\beta}$  line, we get
			\begin{align*}
			\lambda &= \frac{2\times4.02\times10^{-10}}{\sqrt{2^2+0^2+0^2}} sin(40.2/2)\\
			\lambda &= 1.381 {\AA}
		\end{align*}
	With uncertainty, \textbf{$\lambda$ = 1.381 $\pm$ 0.007 \AA}
	Now finally marking both the spectrum with the correct markings
	\begin{center}
	\includegraphics[width=10cm]{a2.png}
	\end{center}
\item The role of the filter is that it shunts away the $K_{\beta}$ line, hence we get only the $K_{\alpha} $ spectrum.
	\end{itemize}

\subsection*{Target:Molybdenum, Crystal:LiF, Filter: None}
We obtain a spectrum like this\\
\begin{center}
	\includegraphics[width=10 cm]{Mo Target, LiF crystal, No Filter.png}\\
	\textit{Spectrum of LiF XRD with Copper Target and no filter}
\end{center}
We try to match it with the table given to us

\begin{center}
	\begin{tabular}{|c|c|c|c|c|}
		\hline
		2 $\theta$ & Intensity & h & k & l \\
		\hline
		17.548 & 95 & 1 & 1 & 1 \\
		\hline
		20.295 & 100 & 2 & 0 & 0 \\
		\hline
		28.843 & 48 & 2 & 2 & 0 \\
		\hline
		33.971 & 10 & 3 & 1 & 1 \\
		\hline
		35.525 & 11 & 2 & 2 & 2 \\
		\hline
		41.251 & 3 & 4 & 0 & 0 \\
		\hline
		45.146 & 4 & 3 & 3 & 1 \\
		\hline
		46.387 & 14 & 4 & 2 & 0 \\
		\hline
		51.119 & 13 & 4 & 2 & 2 \\
		\hline
	\end{tabular}\\
	\textit{2$\theta$ vs Intensity data for Lithium Fluoride}
\end{center}
At the first thought, it seems like the 38.696, 44.996, to an extent 82.998 and 99.628 bands are visible. So we mark that.
\begin{center}
	\includegraphics[width=10 cm]{a1.png}\\
	\textit{Marked spectrum of LiF XRD with Copper Target and no filter}
\end{center}

\subsection*{Target:Molybdenum, Crystal:LiF, Filter:Ni}
We obtain a spectrum like this\\
\begin{center}
	\includegraphics[width=10 cm]{Cu Target, LiF crystal, Ni Filter.png}\\
	\textit{Spectrum of LiF XRD with Copper Target and Ni filter}
\end{center}
This comes as a surprise for us as
\begin{enumerate}
	\item There is no (1 1 1) band or (2 2 2) band in the filtered one
	\item There are bands of only one kind predominant, that is the (h,0,0) type bands. We present the combined two graphs together for a better inference
	\begin{center}
		\includegraphics[width=10 cm]{comb1.png}\\
		\textit{Combined spectrum of LiF XRD and Copper Target, with and without filter}
	\end{center}
\end{enumerate}

\subsection*{Inferences}
\begin{itemize}
	\item The presence of just (h,0,0) bands should imply the crystal lattice is oriented in a single direction, hence it is a mono-crystalline solid.
	\item We can find out the wavelength of the X-Ray with this spectrum.
	We know\\
	\begin{equation*}
		2dsin\theta=n\lambda
	\end{equation*}
	We have 
	\[d=\frac{a}{\sqrt{h^2+k^2+l^2}}\]
	We have a to be 402 pm or 4.02 {\AA}  which is the lattice constant of LiF.
	Calculating the $\lambda$ of the X-Ray, we get
	\begin{align*}
		\lambda &= \frac{2\times4.02\times10^{-10}}{\sqrt{2^2+0^2+0^2}} sin(44.8/2)\\
		\lambda &= 1.531 {\AA}
	\end{align*}
	The uncertainty stems up from the least count of the machine (0.2 degrees). Accounting that\\
	\textbf{$\lambda$ = 1.531 $\pm$ 0.007 \AA}
	\item So the line which we incorrectly marked as (1 1 1) was actually the $K_{\beta}$ line, the major one being the $K_{\alpha}$ line. Doing the similar calculations for the $K_{\beta}$  line, we get
	\begin{align*}
		\lambda &= \frac{2\times4.02\times10^{-10}}{\sqrt{2^2+0^2+0^2}} sin(40.2/2)\\
		\lambda &= 1.381 {\AA}
	\end{align*}
	With uncertainty, \textbf{$\lambda$ = 1.381 $\pm$ 0.007 \AA}
	Now finally marking both the spectrum with the correct markings
	\begin{center}
		\includegraphics[width=10cm]{a2.png}
	\end{center}
	\item The role of the filter is that it shunts away the $K_{\beta}$ line, hence we get only the $K_{\alpha} $ spectrum.
\end{itemize}

\subsection*{Target:Copper, Crystal:Cu, Filter: Ni}

\end{document}