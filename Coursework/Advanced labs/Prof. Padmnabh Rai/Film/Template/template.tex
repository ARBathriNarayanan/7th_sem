\documentclass[]{report}[12 pt]
\usepackage{geometry}
\usepackage{amsmath}
\usepackage{graphicx}
\usepackage{hyperref}
\geometry{margin= 1.5 cm}
\begin{document}
	\begin{titlepage}
	\begin{center}
		\vspace*{1cm}
		
		\Huge
		\textbf{Laboratory Report}
		
		\vspace{0.5cm}
		\LARGE
		Microwave Plasma Chemical Wave Deposition\\
		\vspace{0.5cm}
		\textbf{Guide: Prof. Padmnabh Rai}
		
		\vspace{1.5cm}
		
		\textbf{A R Bathri Narayanan}\\
		Roll no: P0211501\\
		UM DAE Centre for Excellence in Basic Sciences
		
		\vspace{3 cm}
		
		Report presented for the\\
		Advanced Physics Laboratory Course (PL 701)
		
		\vspace{0.8cm}
		
		\includegraphics[width=0.4\textwidth]{cebs.jpg}
		
		\Large
		School of Physical Sciences\\
		UM-DAE Centre for Excellence in Basic Sciences\\
		Mumbai, MH, India\\
		\today
		
	\end{center}
\end{titlepage}
	\section*{Objectives:}
	 Growth and characterization of thin films by electron-beam and thermal evaporation
	systems.
	\begin{enumerate}
	\item Chromium and copper thin film deposition on cover glass slips using electron-beam and
	thermal evaporation techniques, respectively.
	\item Thickness determination of the deposited thin films using a profilometer.
	\item To measure the resistivity of the thin films using the Four-probe or Van der Pauw
		method.
	\end{enumerate}
	\section*{Theory:}
	\begin{enumerate}
		\item  \textbf{Electron Beam Evaporation and Thermal Evaporation:}\\
		Electron Beam Evaporation is a 	form of Physical Vapor Deposition (PVD) in which the target material is bombarded with an electron beam from a charged tungsten filament. From crucible, material evaporates and converts into a gaseous state for deposition of the material to be coated onto the substrate. This is carried out in a high vacuum chamber. Thermal Evaporation is one of the simplest PVD techniques. Basically, target material is heated in a vacuum chamber until its surface
		atoms have sufficient energy to leave its surface. The atoms will traverse the vacuum
		chamber, at thermal energy and coat a substrate. The pressure in the chamber must be below
		the critical point where the mean free path is longer than the distance between the
		evaporation source and the substrate.
		\item \textbf{Vacuum system:}\\
		\textbf{Turbo Molecular Pump (TMP):} It is used to create high vacuum in the
		chamber. The ultimate vacuum of TMP is in the range of $5 \times 10 ^{-10}$ mbar. \\
		\textbf{Rotary pump:} It is a dry pump used to create fore vacuum in the chamber and to serve as a backing pump for the TMP. It achieves an ultimate vacuum in the range of $5.0 \times 10^{-2}$ mbar.\\ 
		\textbf{Substrate heater: }The heater is used to heat the substrate for better deposition. A 2-inch heater is equipped with this 	evaporation system which can be used to heat the substrate up to $800^{\circ}$C.\\ 
		\textbf{Quartz Crystal Microbalance (QCM): }It is used to measure the thickness of the film deposited on a substrate. This is achieved by tracking the frequency response of a quartz crystal during the coating
		process. The change in frequency can be directly related to the amount of coating material on
		the crystal surface.
		\item \textbf{Thickness Profilometer:}\\
		A thickness profilometer is an essential instrument used for
		measuring the thickness of thin films and coatings. It works by scanning the surface of a
		sample and recording the topographical variations with high precision. The device typically uses a stylus or optical method to trace the surface contours, providing detailed information 	about the film uniformity and thickness.
		
	\end{enumerate}
	\section*{Observations:}
	\begin{center}

	\end{center}

\end{document}