\documentclass[]{report}[12 pt]
\usepackage{geometry}
\usepackage{amsmath}
\usepackage{graphicx}
\usepackage{hyperref}
\usepackage{multirow}
\geometry{margin= 1.5 cm}

\begin{document}
	\begin{titlepage}
	\begin{center}
		\vspace*{1cm}
		
		\Huge
		\textbf{Laboratory Report}
		
		\vspace{0.5cm}
		\LARGE
		Microwave Plasma Chemical Wave Deposition\\
		\vspace{0.5cm}
		\textbf{Guide: Prof. Padmnabh Rai}
		
		\vspace{1.5cm}
		
		\textbf{A R Bathri Narayanan}\\
		Roll no: P0211501\\
		UM DAE Centre for Excellence in Basic Sciences
		
		\vspace{3 cm}
		
		Report presented for the\\
		Advanced Physics Laboratory Course (PL 701)
		
		\vspace{0.8cm}
		
		\includegraphics[width=0.4\textwidth]{cebs.jpg}
		
		\Large
		School of Physical Sciences\\
		UM-DAE Centre for Excellence in Basic Sciences\\
		Mumbai, MH, India\\
		\today
		
	\end{center}
\end{titlepage}
	\section*{Objectives:}
	 Growth and characterization of thin films by electron-beam and thermal evaporation
	systems.
	\begin{enumerate}
	\item Chromium and copper thin film deposition on cover glass slips using electron-beam and
	thermal evaporation techniques, respectively.
	\item Thickness determination of the deposited thin films using a profilometer.
	\end{enumerate}
	\section*{Theory:}
	\begin{enumerate}
		\item  \textbf{Electron Beam Evaporation and Thermal Evaporation:}\\
		Electron Beam Evaporation is a 	form of Physical Vapor Deposition (PVD) in which the target material is bombarded with an electron beam from a charged tungsten filament. From crucible, material evaporates and converts into a gaseous state for deposition of the material to be coated onto the substrate. This is carried out in a high vacuum chamber. Thermal Evaporation is one of the simplest PVD techniques. Basically, target material is heated in a vacuum chamber until its surface
		atoms have sufficient energy to leave its surface. The atoms will traverse the vacuum
		chamber, at thermal energy and coat a substrate. The pressure in the chamber must be below
		the critical point where the mean free path is longer than the distance between the
		evaporation source and the substrate.
		\item \textbf{Vacuum system:}\\
		\textbf{Turbo Molecular Pump (TMP):} It is used to create high vacuum in the
		chamber. The ultimate vacuum of TMP is in the range of $5 \times 10 ^{-10}$ mbar. \\
		\textbf{Rotary pump:} It is a dry pump used to create fore vacuum in the chamber and to serve as a backing pump for the TMP. It achieves an ultimate vacuum in the range of $5.0 \times 10^{-2}$ mbar.\\ 
		\textbf{Substrate heater: }The heater is used to heat the substrate for better deposition. A 2-inch heater is equipped with this 	evaporation system which can be used to heat the substrate up to $800^{\circ}$C.\\ 
		\textbf{Quartz Crystal Microbalance (QCM): }It is used to measure the thickness of the film deposited on a substrate. This is achieved by tracking the frequency response of a quartz crystal during the coating
		process. The change in frequency can be directly related to the amount of coating material on
		the crystal surface.
		\item \textbf{Thickness Profilometer:}\\
		A thickness profilometer is an essential instrument used for
		measuring the thickness of thin films and coatings. It works by scanning the surface of a
		sample and recording the topographical variations with high precision. The device typically uses a stylus or optical method to trace the surface contours, providing detailed information 	about the film uniformity and thickness.
		
	\end{enumerate}
	\section*{Observations:}
	\subsection*{Electron beam evaporation}
\begin{center}
	\begin{tabular}{|c|cccccc|}
		\hline
		\multirow{2}{*}{\begin{tabular}[c]{@{}c@{}}Base Vaccum \\ (mBar)\end{tabular}} & \multicolumn{6}{c|}{Evaporation beam condition}                                                                                                                                                                                                                                                                                                                                                                                                                                                               \\ \cline{2-7} 
		& \multicolumn{1}{c|}{\begin{tabular}[c]{@{}c@{}}Source\\ Material\end{tabular}} & \multicolumn{1}{c|}{\begin{tabular}[c]{@{}c@{}}Emission\\ Current(mA)\end{tabular}} & \multicolumn{1}{c|}{\begin{tabular}[c]{@{}c@{}}Filament current \\ (A)\end{tabular}} & \multicolumn{1}{c|}{\begin{tabular}[c]{@{}c@{}}DC Output\\ Voltage(kV)\end{tabular}} & \multicolumn{1}{c|}{\begin{tabular}[c]{@{}c@{}}Deposition \\ rate (nm/s)\end{tabular}} & \begin{tabular}[c]{@{}c@{}}Thickness by\\ QCM (nm)\end{tabular} \\ \hline
		(8.8      $\pm$ 0.1) $\times 10^{-7}$                                                                   & \multicolumn{1}{c|}{Cr}                                                        & \multicolumn{1}{c|}{12 $\pm$ 1}                                                             & \multicolumn{1}{c|}{14.5 $\pm$ 0.1}                                                            & \multicolumn{1}{c|}{7.36 $\pm$ 0.01}                                                            & \multicolumn{1}{c|}{0.40 $\pm$ 0.01}                                                              & 5 $\pm$ 0.8                                                           \\ \hline
	\end{tabular}
\end{center}

	\subsection*{Thermal evaporation}
\begin{center}
		\begin{tabular}{|c|cclccc|}
			\hline
			\multirow{2}{*}{\begin{tabular}[c]{@{}c@{}}Base Vaccum \\ (mBar)\end{tabular}} & \multicolumn{6}{c|}{Evaporation beam condition}                                                                                                                                                                                                                                                                                                                                                                \\ \cline{2-7} 
			& \multicolumn{1}{c|}{\begin{tabular}[c]{@{}c@{}}Source\\ Material\end{tabular}} & \multicolumn{2}{c|}{\begin{tabular}[c]{@{}c@{}}Emission current \\ (A)\end{tabular}} & \multicolumn{1}{c|}{\begin{tabular}[c]{@{}c@{}}Voltage\\ (kV)\end{tabular}} & \multicolumn{1}{c|}{\begin{tabular}[c]{@{}c@{}}Deposition \\ rate (nm/s)\end{tabular}} & \begin{tabular}[c]{@{}c@{}}Thickness by\\ QCM (nm)\end{tabular} \\ \hline
			(1.2  $\pm$0.1) $\times10^{-6}$                                                                         & \multicolumn{1}{c|}{Cu}                                                        & \multicolumn{2}{c|}{112.0 $\pm$ 1.5}                                                           & \multicolumn{1}{c|}{40.5 $\pm$ 1}                                                   & \multicolumn{1}{c|}{0.65 $\pm$ 0.05}                                                              & 45 $\pm$ 1.3                                                             \\ \hline
		\end{tabular}
\end{center}

\subsection*{Calibration factor}

\[ \text{Calibration factor (C.F)}= \frac{\text{Thickness obtained in the Profilometer}}{\text{Thickness expected}}\]

Thickness obtained in the Profilometer = $54 \pm 1 $ nm

\[ \text{Calibration factor (C.F)}= \frac{54}{50} = 1.08\]
\[\text{Error in Calibration factor }= \frac{\delta a}{a}+\frac{\delta b + \delta c}{b+c}\]
Where a is measurement relating to profilometer, b is measurement relating to electron beam and c is measurement relating to thermal evaporation.
\[\delta (C.F) =  \frac{1}{54}+\frac{0.8 + 1.3}{50}\]
\[\delta (C.F) =  0.06\]

\section*{Results}
\begin{itemize}
	\item We have successfully made a film of Chromium-Copper deposition using Electron beam and Thermal evaporation techniques.
	\item We have used $5\pm 0.8$ nm thickness for Chromium and $45 \pm 1.3$ nm for Copper to give roughly $50 \pm 2.1$ nm thickness of film. 
	\item The calibration factor that can be used for further studies is $1.08 \pm 0.06$ 
	\item \textbf{Note: }The errors for thickness in electron beam was done by measuring the time taken to close the shutter and reaction time, multiplied by the rate of deposition.
\end{itemize}
\end{document}