\documentclass[11pt]{beamer}
\usepackage[utf8]{inputenc}
\usepackage[T1]{fontenc}
\usepackage{lmodern}
\usepackage[english]{babel}
\usetheme{AnnArbor}
\begin{document}
	\author{A R Bathri Narayanan}
	\title{The Gaussian Beam}
	\subtitle{Presentation for\\Femtosecond and Attosecond Pulses (P-704)}
	%\logo{}
	\institute{UM DAE Centre for Excellence in Basic Sciences}
	\date{November 25, 2024}
	\subject{Femtosecond and Attosecond pulses}
	%\setbeamercovered{transparent}
	%\setbeamertemplate{navigation symbols}{}
	\begin{frame}[plain]
		\maketitle
	\end{frame}
	
	\begin{frame}
		\frametitle{Prelude}
		A paraxial wave is a plane wave travelling along the z direction ($e^{-ikz}$) with wavenumber $\frac{2\pi}{\lambda}$ for wavelength $\lambda$, modulated by a complex envelope A(\textbf{r}), being a slowly varying function of position.  The complex amplitude is 
		\[U(\textbf{r})=A(\textbf{r})e^{-ikz}\]
		The envelope is taken to be approximately constant within a
		neighborhood of size $\lambda$, so that the wave locally maintains its plane-
		wave nature but exhibits wavefront normals that are paraxial rays.
	\end{frame}
	
	\begin{frame}
		\frametitle{The Gaussian Solution}
	\end{frame}
	
	\begin{frame}{Properties of the Gaussian Beam}
		
	\end{frame}
	
	\begin{frame}{Intensity}
		
	\end{frame}
	
	\begin{frame}{Intensity}
		
	\end{frame}
	
	\begin{frame}{Power}
		
	\end{frame}
	
	\begin{frame}{Power}
		
	\end{frame}
	
	\begin{frame}{Beam Width}
		
	\end{frame}
	
	\begin{frame}{Beam Divergence}
		
	\end{frame}
	
	\begin{frame}{Depth Focus}
		
	\end{frame}
	
	\begin{frame}{Phase}
		
	\end{frame}
	
	\begin{frame}{Wavefronts}
		
	\end{frame}
	
	\begin{frame}{Wavefronts}
		
	\end{frame}
	
	\begin{frame}{Characterisation of Gaussian Beam}
		
	\end{frame}
	
	\begin{frame}{To Summarize}
		
	\end{frame}
	
	\begin{frame}{Beam Quality}
		
	\end{frame}
	
	\begin{frame}{Beam Quality}
		
	\end{frame}
	
	\begin{frame}{Take-Home Messages}
		
	\end{frame}
	
	\begin{frame}
		\begin{center}
		\LARGE Thank You!!
		\end{center}
	
    \end{frame}
	
	
\end{document}